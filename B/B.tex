\subsection{\texorpdfstring{\\
}{ }}\label{section}

\section{}\label{section-1}

\section{B. Deuxième année - La
technique.}\label{b.-deuxiuxe8me-annuxe9e---la-technique.}

\subsection{}\label{section-2}

\subsection{\texorpdfstring{\\
}{ }}\label{section-3}





\subsection{B03 -- Serrage dans le coin par 3
Bandes}\label{b03-serrage-dans-le-coin-par-3-bandes}


\subsection{B04 --  -
}\label{b04-3-billes-dans-le-coin-par-le-tour--}



\subsection{B05 - }\label{b05---3-bandes-dans-le-carruxe9}


\subsection{B06 --  -
}\label{b06-rappel-dans-le-coin-par-une-bande--}


\subsection{B07 --
}\label{b07-rappels-extruxeames-par-une-bande}



\subsection{B08 -- Rappel à la petite bande}\label{b08-rappel-uxe0-la-petite-bande}



\subsection{B09 -- Rappel direct ou par la bande
?}\label{b09-rappel-direct-ou-par-la-bande}



\subsection{B11 - Le rétro qui double
}\label{b11---le-ruxe9tro-qui-double}

C'est un rétro violent qui faut utiliser avec discernement. Avec un
rétro normal, nous sentons que la 2 nous échappe, elle s'écarte ou bien
nous ne parviendrons pas à jouer suffisamment doucement pour à la fois
faire le point et ramener et garder la 2 près de la 3. Avant de choisir
ce coup, vérifions tout de même que la 2 ne s'échappe pas trop loin....
En fait, celle-ci rentre presque sinon, la doubler nous la fera perdre
définitivement.

Prendre la 2 en rétro pas trop bas en force, afin de donner un mouvement
de retour relativement lent à la 1 pendant que la 2 prend l'énergie
suffisante pour parcourir 4 fois la largeur de la table. On espère que
la 2 reviendra dans le jeu, l'effet appliqué, contraire à sa fuite,
devant la forcer à y rester.

Remarque : Il sera prudent de vérifier que la 2 ne vient pas percuter la
1 avant de doubler, auquel cas, le résultat deviendrait aléatoire. Si
tel devait être le cas, on aurait dû choisir un rétro simple, profitant
de la bosse pour ralentir la 2 et conserver une chance de regroupement.
Si la 2 rentre difficilement dans le jeu même après deux retours, il
sera prudent avant toute décision, de vérifier si un B simple sur la 3
n'est pas possible en attendant une position plus favorable.

Réfléchissons avant de tenter ce coup... il semble facile quand on est
bien stable et horizontal mais la rentrée convenable est souvent
imprécise. Ce point constitue souvent un point d'attente et non de
construction.

\subsection{B12 - Le coulé qui
double}\label{b12---le-couluxe9-qui-double}

Les billes sont disposées de telle sorte qu'un direct est souvent
possible. Cependant, si nous avons la chance d'avoir la 1 en position
dominante, un coulé qui double ramène les billes dans le chapeau.

Figure 1 : Un coulé simple est possible mais nous risquons de rester au
milieu du jeu avec la 2 dans notre dos. Appliquons donc un coulé pas
trop haut pour donner une bonne énergie à la 2 avec effet permettant la
rentrée ici, effet à droite. La 1 voyage assez lentement et s'arrête à
la 3, rejointe bientôt par la 2. Le seul danger est le risque de contre
de la 2 sur la 1 si l'inclinaison 1-2 sur la petite bande est trop
importante. Afin d'éviter ce danger, on peut viser plus bas sur la 1 et
jouer plus durement, obligeant la 2 à passer avant la 1. Cependant, je
le déconseille surtout aux non avertis car on risque en même temps de
jouer trop court avec la 1 et trop long avec la 2 ! On peut aussi jouer
plus souplement, un rien plus fin sur la 2. On augmente ainsi la vitesse
de la 1 ce qui lui permet de passer la première. On réalise alors le
point en touchant éventuellement la bande devant la 3 et la 2 rentre
moins bien. Enfin, on peut aussi mettre l'effet contraire et la 2 alors
s'échappe un peu du jeu mais reste accessible.

Figure 2 : Cas particulier : si la 2 et la 1 sont alignées le long d'une
bande, coulez directement avec effet côté extérieur à la bande proche.
La 2 revient en « s'écrasant » à la bande. La 1 passe à côté du retour
de la 2 pour aller faire le point. La 2 revient par doublé... une
merveille !

Entraînement : Ce point est un exercice magnifique pour la maîtrise du
jeu. C'est le moment de coupler le coulé qui double au rétro qui double.
Figure 3 : J'ai figuré les différentes positions des billes 2 et 3, la 3
voyageant le long de la petite bande et la 2 voyageant le long de la
première ligne de cadre. Placer successivement la 2 en A, A' et A'' avec
la 3 en position B. Ensuite recommencer avec les mêmes positions de la 2
après avoir placé la 3 en B' et enfin encore une série avec la 3 en B''.
Vous constaterez vous-mêmes au cours de l'exercice que vous passez du
rétro au coulé et inversement et que vous allez être amenés à viser la 1
plus près du centre que d'habitude si vous essayez de bien rentrer les
billes dans le chapeau. Cet exercice devient l'antichambre de l'amorti
qui sera repris plus amplement en quatrième année. Avec des séries de 5
essais sur chaque emplacement, exigez 100 \% de réussite ! Cela
demandera quelques heures, le temps de mettre la mécanique dans les
mains mais ça payera. Vous vous servirez de cette maîtrise acquise en de
nombreuses occasions.

\subsection{B13 -- La bricole simple}\label{b13-la-bricole-simple}

La bricole, même simple est un point angoisse FL pour beaucoup de
joueurs. Cependant le calcul est facile même si la figure paraît
compliquée.

Figure 1 : Les billes sont à peu près à la même distance de la bande
choisie pour tenter le coup, une finesse étant trop difficile ou
impossible. Par le centre de la 1, abaisser une perpendiculaire sur la
bande face aux billes 2 et 3, la coupant au point A. Du milieu du
segment 2-3, abaisser une perpendiculaire sur la même bande la coupant
au point B. Le milieu du segment A-B, soit le point R, -

constitue le point de repère rapproché, soit le point R. Il ne nous
reste plus qu'à viser ce point sans effet R* et de préférence bille
en-tête pour pallier à une e éventuelle mini erreur de prise sans effet.

Figure 2 : Ce serait trop beau si les bricoles à réaliser étaient
toujours aussi bien disposées. Ici, la 1 n'est pas à la même distance
que les billes 2 et 3. Il conviendra donc d'appliquer une correction au
calcul. Procédons d'abord comme expliqué pour la figure 1. Abaissons les
perpendiculaires pour déterminer les points A, B et R. Si nous visons ce
repère R, nous obtiendrions une trajectoire erronée soit la droite «x».
Traçons une parallèle à la grande bande passant par les centres des
billes 2 et 3 coupant la droite x en un point soit E. Abaissons la
perpendiculaire du point E vers la bande soit le point d'intersection C.
On a ainsi reporté sur la bande, l'erreur commise par notre construction
initiale. Le milieu du segment R-C devient notre point de repère
rapproché Rc (repère corrigé).

Remarque : L'habileté du bricoleur consiste à bien fixer le point de
repère à viser. Comme il n'est pas autorisé d'utiliser les règles et les
équerres, outils bien utiles en l'occurrence, la réussite réside dans la
faculté d'imaginer dans l'espace le tracé expliqué plus haut. Il ne faut
pas hésiter à utiliser sa canne afin de projeter des directions
imaginaires seulement perçues par le jo verra que des arabesques et
autres démonstrations spectaculaires mais nous, on s'en fiche ! La
figure 2 comprend plus d'éléments que ceux réellement utilisés, ceci
afin de montrer une construction complète pour les géomètres exigeants.
La réalisation « sur le terrain » est plus aisée que sur le papier.
Quand on aura acquis un peu d'habitude à l'entraînement, on pourra
aisément se passer des intermédiaires. On se placera à la bande face aux
billes pour en apprécier directement l'emplacement du point de repère
rapproché même avec correction incorporée. Enfin, si les billes 2 et 3
sont un peu écartées ou inclinées, c'est au joueur d'apprécier l'endroit
où il doit toucher pour réaliser et par là faire varier l'emplacement du
repère.

\subsection{B14 - Bricoles avec
effets}\label{b14---bricoles-avec-effets}

Les bricoles avec effet sont très délicates. Elles demandent une bonne
maîtrise. Le jugement en est difficile et la réalisation tout autant. En
voici quelques exemples.

Figure 1 : On peut jouer autrement qu'en bricole, par exemple en rétro
direct ou en 1B. Cependant la 2 risque de ne pas revenir si
l'inclinaison 1-2 sur la grande bande est trop importante. Dès lors, la
bricole effet contraire devient la meilleure solution car la 2 s'écarte
peu de la grande bande et la 1, mue de son effet contraire, s'arrête en
face ce qui permet un rétro sur la longueur et de prendre la dominance.

Figure 2 : Nous sommes masqués. En attendant de connaître le massé,
jouons « en face » avec bon effet. C'est un point d'attente. Nous
restons sur le billard. Si 2 et 3 sont l'une très près de l'autre on
peut aussi bricoler en face de 2 et 3 avec effet contraire. Le point est
à juger sur place mais attention! L'effet haut de bille est généralement
difficile à jauger. L'angle de retour dépend de l'impulsion donnée, de
l'imprégnation opérée par le joueur et du coefficient « d'accroche » du
tapis de la bande. Bon nombre d'échecs vous guettent...

Figure 3 : La bricole deux bandes face à un coin est une proche parente
du 2B traditionnel. Les billes 2 et 3 sont proches de l'axe du coin et
la 1 est en face à la même distance. Celle-ci est donc symétrique à
l'impact espéré sur les billes 2 et 3 par rapport à la bissectrice.
Jouez haut de bille pour opérer une parallèle à l'axe avec du bon effet.
Attention, cette méthode n'est vraiment valable qu'au voisinage de l'axe
(10 cm.). Elle crée une petite erreur qui dépasse une demie bille dès
qu'on s'écarte.

Figure 4 : La 1 n'est plus symétriquement disposée. Une autre solution
est parfois possible comme un 1B sinon, en désespoir de cause, on peut
encore tenter un 2 bandes bricolé mais cela devient très délicat. Un
truc est de se placer face au coin, de repérer le milieu du segment {[}1
- 2/3{]}, soit le point X. Tracez mentalement une droite passant par le
coin et X. Ensuite appliquez une parallèle bon effet à cette droite
imaginée feil-coin{]} ... très délicat... à ne sortir qu'en cas de
dernière nécessité ou en cas de jour de grâce...

Remarque : Toutes ces positions sont beaucoup plus délicates qu'il n'y
paraît. La réussite dépendra non seulement de l'expérience et de la
maîtrise du joueur mais aussi de l'état et des caractéristiques de la
table sur laquelle on s'exerce. Un entraînement soutenu permettra de
s'habituer à bien juger et à posséder une fois pour toute la touche
adéquate. Bon travail !

\subsection{B15 - Bricoles deux et trois bandes -
}\label{b15---bricoles-deux-et-trois-bandes--}

Ce sont deux bricoles différentes mais bien utiles et si souvent
négligées au profit d'un 1B qui laisse la 1 au milieu des deux autres
billes.

Figure 1 : On s'est empêtré à la petite bande en tentant de rentrer les
billes 2 et 3. Un petit 1B ne serait pas une mauvaise solution mais elle
impose de nouveau un rétro 1B à suivre et laisse la 1 avec une prise de
dominante difficile. Tentez donc une bricole 2B comme dessiné bille en
tête sans effet et viser pour toucher la 2°bande juste devant la 2. La 1
aura tendance à prendre la dominante en un seul coup. Si par malheur ce
n'est pas parfaitement réussi, la situation ne sera pas détériorée par
rapport à la situation précédente.

Figure 2 : La position de la 3 est telle que son rappel est très
difficile voire impossible. Un 1B sur la 3 donne un résultat très
aléatoire. On devrait jouer assez fort et la 3 reviendrait au coin
pendant que la 2 change de côté. Un 1B sur la 2 est sans doute le point
le plus facile mais il donne une position très inconfortable. Serrez
donc en un coup. Si la 2 n'est pas trop éloignée de la petite bande,
appliquez une bricole deux bandes par l'arrière, mi-hauteur ou mi-haute,
assez grosse et bon effet bien dosé. Les billes se retrouvent dans le
chapeau au croisillon du cadre. Au deuxième coup, vous aurez l'embarras
du choix pour « tourner » le jeu.

Remarques : - Une prise fine amène la 1 entre la petite bande et la
bissectrice.

- Une prise moyenne amène vers la bissectrice.

Une prise grosse amène aussi vers la bissectrice avec une tendance à la
dépasser mais attention : pas beaucoup. Pour ramener la 1 plus près de
la grande bande, il conviendra de

bricoler bille basse avec un bon effet maximum, presque coup du rétro...
- Le point est plus facile lorsque la 3 est un peu plus près de la
petite bande que la grande...

Notez bien : quel que soit l'essai, la position suivante sera presque
toujours favorable au jeu constructif. Ceci est un point typique du jeu
de cadre. Il demande une petite habitude pour jauger de la grosseur de
bille et l'évitement de la bosse (pas toujours défavorable) de la 1 sur
la 2 après 3 bandes...

\subsection{B16 - Rétro deux bandes}\label{b16---ruxe9tro-deux-bandes}

Le rétro deux bandes peut sauver des situations qui se dégradent.

Comme le montre la figure, la 1 est face au coin et le jeu est en
expansion, càd qu'il s'éloigne vers le milieu de la table. Aucun rétro
direct n'assure une rentrée convenable.

Dans certains cas un 1B sans effet vers la grande bande voisine peut
assurer une rentrée de la 2 mais le coup est assez délicat avec ce que
nous sommes supposés déjà maîtriser. Nous le garderons en mémoire pour
l'appliquer éventuellement plus tard.

L'exercice du jour consiste à s'habituer à appliquer un rétro 2 bandes,
bille basse, assez grosse, effet favorable. La principale difficulté
réside dans le choix du point de repère rapproché sur la petite bande.
D'une manière générale, prenons l'habitude de l'évaluer au tiers de la
distance 3-grande bande à abaisser sur la petite bande lorsque les
billes 2 et 3 sont à peu de chose près à la même distance de cette
petite bande et que nous tentons un rétro « normal ». La bonne
estimation du point d'impact de la 1 sur la petite bande est important.
La bille possède un effet important qui rend l'angle de réflexion assez
évasé. Il sera prudent de tester avant de consommer ! Les tables et les
tapis possèdent des « rendus » différents. Dès que nous avons bien
repéré le point d'impact de la 1 à la petite bande, dans notre tête,
nous avons transformé un 2B en un 1B ! Si les billes 2 et 3 ne sont pas
parallèles à la petite bande, il faudra bien estimer le point de repère
rapproché, à l'œil. Enfin, ce point de rappel donne rarement un serrage
et encore moins souvent une position dominante mais il ne dégrade pas la
situation et permet souvent un positionnement favorable à la
construction.

Remarques : - le même point est jouable sur la grande bande, donc sur le
côté.

Un deux bandes peut être appliqué « à l'envers » càd sur la 3 vers la
grande bande opposée. Essayez-le comme dessert. Il est moins difficile
qu'il y paraît et il rentre la 3. (voir en pointillé sur la figure).

\subsection{B17 - Rétro bille-bande effet
rentrant}\label{b17---ruxe9tro-bille-bande-effet-rentrant}

Voici deux exemples pour vous permettre d'un peu respirer. Les plaques
précédentes sont un peu lourdes et demandent beaucoup de travail. Ici,
nous exposons des cas de « finesse » de jeu. On voit très rarement ces
points joués de cette façon et pourtant dans certains cas, ils seront
très utiles.

Figure 1 : Il s'agit d'un rétro bille très basse, souple et effet
contraire sur la petite bande avec une tenue de canne légère. C'est un
point assez sophistiqué. Il est dangereux car il est difficile de
maîtriser l'effet contraire. L'exercice nous aidera au moins pour cela.
L'effet contraire imprime à la 2 un effet inverse la rabaissant vers la
bande. La 1 venant se coincer entre la 3 et la bande offre un barrage au
retour de la 2. Il n'est pas rare que la nouvelle position offre une
possibilité d'américaine. Il est également prisé dans le jeu de 1B. On
aurait pu jouer un rétro direct ce qui est une bonne solution aussi mais
le retour de la 2 le long de la bande doit s'arrêter à la 3. Trop court,
le jeu de rappel continue et trop long, on risque d'être masqué !

Figure 2 : C'est plus un rappel qu'une nouveauté. Lorsque vous aurez
pris l'habitude de manipuler l'effet contraire sur la bande proche, ce
coup vous deviendra familier et utile. Constatez que la direction 1-2 «
coupe » la grande bande assez haut sur la table càd avec un angle très
petit. Un rétro direct effet contraire ne rentre pas la 3 mais l'éloigne
peu. C'est donc bien jouer. Cependant, si vous appliquez la théorie de
la figure 1 en étendant le coup pour toucher la grande bande devant la
3, vous serrez d'autant le point et la 3 rentre plus près et même dans
certains cas, carrément dans le chapeau. Donc même principe : rétro
bille basse, souple, effet contraire et canne légère.

Remarque : on appliquera cette technique uniquement dans le but de
serrer le jeu et à condition de se retrouver sur sa « bonne » main après
la réalisation. La notion de bonne main reviendra plus tard. Beaucoup
ont une bonne main et une autre main plus que mauvaise, carrément
minable. Vous aurez toujours une meilleure main que l'autre mais parfois
il est triste de devoir la passer avec une position facile mais « mal
mise ». Essayez donc de jouer de temps en temps avec votre « mauvaise »
main. Elle ne deviendra pas meilleure que l'autre mais au moins, vous ne
la perdrez plus face à une position facile. Trop souvent, un joueur
possède une mauvaise main parce qu'il ne s'en sert qu'en match !

\subsection{B18 -- Rappel par le 8}\label{b18-rappel-par-le-8}

Le rappel par le 8 est une appellation amusante inspirée du dessin que
tracent les billes 1 et 2. Ce point est très spectaculaire et il attire
l'admiration du spectateur. Joué fort, avec les billes qui roulent et
voilà la 2 qui rentre doucement : ça épate. Cependant, le coup est d'une
difficulté très moyenne pour qui s'est un peu entraîné.

C'est un rétro violent, effet contraire et très soutenu. La 1 revient à
petite vitesse sur la grande bande arrière. Elle reprend vigueur à son
contact grâce à son effet puissant et assure tranquillement le chemin à
parcourir en une ou deux bandes jusqu'à la 3. Quant à la 2, son
cheminement paraît curieux bien que logique. Suivant le positionnement
1-2, elle sera projetée violemment sur la grande bande opposée entre les
mouches 1 et 3, soit à peu près à une distance de 30 à 90 cm du coin
supérieur, soit le point d'impact 11. Vu l'effet contraire induit, sa
course se redresse càd qu'elle se rapproche de la perpendiculaire à la
grande bande en 11. Elle coupe la petite bande au point 12. Si on a
frappé fort assez, elle conserve encore de l'effet contraire à ce moment
mais va le perdre après l'impact sur la grande bande soit 13. La voilà
nantie d'un nouvel effet qui devient contraire en retrouvant la première
grande bande, faible il est vrai, mais suffisant pour un dernier
redressement vers le chapeau.

Remarque : - Avant de tenter l'essai, on vérifiera que le point 12 est
bien situé sur la petite bande moitié gauche comme la figure le montre
(moitié droite si les billes sont placées de l'autre côté). Si jamais le
point 12 devait être situé sur la grande bande, la 2 ne reviendrait
jamais !

- Il est difficile d'amener les 3 billes dans le chapeau à chaque coup
mais la nouvelle position

gagnée sera toujours plus favorable que la précédente.

Entraînement: Exercez-vous à placer les billes 1 et 2 à différentes
hauteurs, en restant de préférence dans la moitié inférieure de la table
avec différentes inclinaisons. Cela vous habituera non seulement à la
justesse du coup mais aussi à juger des frontières à ne pas dépasser.
Enfin, on peut aussi monter les billes 1 et 2 au-dessus du milieu de la
table mais la rentrée deviendra de plus en plus difficile au fur et à
mesure que vous vous approchez de la petite bande supérieure. Vous
trouverez cet entraînement instructif et divertissant..

\subsection{B19- Rappel par rétro
amorti}\label{b19--rappel-par-ruxe9tro-amorti}

Voici deux exemples d'amortis avec les deux plaquettes B19 et B20. C'est
un avant-goût de ce qui vous attend en quatrième année.

Figure 1 : Nous avons vu en première année, les points de visée qui
peuvent être précisément déterminés sur la bille 1. Une bille fait un
peu plus de 6 cm. de diamètre. Pour que la flèche ne glisse pas, nous
sommes amenés à la frapper dans un cercle inscrit au centre bille de
environ 3 cm. de diamètre. Imaginez un cercle au centre bille de
seulement 1 cm de diamètre. Ramenez mentalement tous les points de
visées déjà connus vers le centre bille en suivant les rayons par
lesquels ils passent et arrêtez ces points dès qu'ils sont dans ce
fameux cercle de 1 cm de diamètre. Les points de visées dessinent la
même figure mais celle-ci est plus petite. Vous avez maintenant de quoi
jouer droit, coulé ou rétro de la même façon mais pour récolter les
mêmes résultats, vous devrez jouer plus gros, plus souple, plus fluide,
sans à-coup. Vous devrez pénétrer (!) au lieu de frapper. La 2 aura une
direction plus proche de celle de la 1 dans son premier tronçon, qui,
elle, aura un mouvement plus lent car la 2 aura récolté la grosse part
de l'énergie donnée. Enfin la 2, avec une déviation moins grande, moins
éclatée, « rentrera » des positions jusqu'ici «incontrôlables »,

Figure 2 : C'est un angle droit incliné à 45°. L'exécution d'un rétro
habituel tel que nous l'avons vu, ramènerait la 2 au mieux, au milieu de
la petite bande pendant que la 1 et la 3 seraient à la grande bande.
Prenez donc la 2 pleine moins un petit morceau (un chêche) et visez,
comme pour un angle droit, le centre du secteur gauche inférieur,
attention, celui dans le cercle de 1 cm. au centre bille. Si vous
parvenez à bien vous abandonner dans votre acte, faisant confiance à
votre matériel, càd en ne forçant pas votre volonté de domination dans
votre coup, vous serez étonné du résultat, d'autant plus que cela aide à
la concentration et que vous allez avoir l'impression que vous et votre
canne ne faites plus qu'un ! Vous aurez l'impression que vous avez
acquis le pouvoir de décider que les billes se positionnent là où vous
désirez qu'elles soient. C'est grisant !

Remarque : L'abandon de soi est chose difficile. C'est une abnégation de
sa propre personne. C'est faire confiance sans que nous ne puissions
intervenir même si nous savons que c'est nous qui l'avons produit. C'est
« ne rien mettre dans sa canne ». Ne mettre aucune intention, faire le
vide, ... ou toute autre expression décrivant un état de dépendance nous
est naturellement insupportable. Nous nous sentons à la merci du premier
venu... Pourtant, désolé pour les joueurs qui n'accepteront pas ou qui
ne parviendront pas à l'accepter, il n'y aura pas d'accès aux catégories
supérieures pour les rebelles...

\subsection{B20 - Coulé amorti rappel par 3 bandes --
}\label{b20---couluxe9-amorti-rappel-par-3-bandes}

Ce point demande non seulement une bonne maîtrise du coulé mais aussi de
l'amorti. Bien que nous reverrons l'amorti en quatrième année, il est
intéressant de nous tester maintenant sur nos facultés à nous « détacher
» de nous-mêmes. Comme pour la plaquette précédente, nous devons nous «
identifier \textgreater{}\textgreater{} à notre matériel... Ce n'est pas
moi qui joue, c'est ma canne... !?...

Voyons l'exemple de la figure. Contrairement à ce qu'on pourrait croire,
nous n'allons pas d'abord regarder ni la grosseur ni la hauteur de bille
à prendre pour réussir le point !

examiner quelle grosseur de la 2 doit être prise pour que celle-ci fasse
le tour de la table et vienne se positionner dans le « chapeau » lorsque
nous la prendrons en coulé. Quand nous avons bien fixé cette grosseur,
voyons si, en touchant la 2 de cette manière, nous réalisons le point.
Si oui, ne changez rien et coulez normalement, peut-être avec une prise
pas trop haute de manière à ralentir la 1, qu'elle ne percute pas la 3
pour la faire se sauver !... Si non, ne faites pas varier cette grosseur
de bille mais bien la distance entre le point de touche sur la 1 et son
centre. Vous allez ainsi faire varier l'angle et la vitesse de la 1.
Vous constaterez rapidement que vous entrez dans « la famille » des
amortis car vous vous rapprochez du centre bille... Maintenant, tout
devient mental ! Vous devez faire le vide autour de vous. Vous entrez
dans votre canne et dans un élan harmonieux, vous pénétrez dans la 2
comme si vous y enchâssiez votre procédé... Le coup porté n'est qu'un
limage plus long et non une frappe ! Vous vous sentez transporté avec la
1. Le jeu est à vous et les billes vont où vous désirez qu'elles
aillent... On rêve...

Remarque :

Par extension, on comprendra que ce coup est d'une importance capitale.
Il ne sert pas seulement à faire tourner la 2. Il peut être employé pour
la faire « rentrer » en une ou deux bandes dans des positions que nous
avons l'habitude de faire en coups droits, coulés ou rétros normaux et
qui ne suffiraient pas pour une rentrée. L'effet ici induit grâce à la
pénétration est d'une telle importance qu'il permet des rentrées
apparemment impossibles. Nous en verrons plusieurs applications plus
tard. Quoi qu'en disent ceux qui savent ou qui croient savoir, ce coup
est difficile. A ce stade de notre étude, il ne sera pas nécessaire
d'être parfait mais si le simple rétro rentrant par trois bandes et le
coulé rentrant par trois bandes pouvaient être acquis, cela nous
facilitera la tâche.

\subsection{B21 - Long rétro, petit
déplacement}\label{b21---long-ruxe9tro-petit-duxe9placement}

Cette position fait partie des tentations dangereuses. Il serait si
rassurant de faire un petit direct et d'engranger déjà un point. C'est
vrai, mais il est trompeur. L'angle 1-3-2 est évasé et un simple direct
n'est pas si simple. Pour peu qu'on soit obligé de forcer ou de tenter
un 90°, nous resterions avec un point d'attente, au mieux, un coup de
rappel qui nous laisserait encore au milieu de la table. Beaucoup de
joueurs tentent même un 1B en touchant d'abord la 3. Quel
désappointement lorsque leur bille rate l'arrière de la 2, laissant de
surcroît, une position dominante à l'adversaire.

Il est temps de commencer à penser tactique. * Pousser les billes devant
soi les révèle. C'est déjà bien.

☺ Les pousser à la grande bande en plus, c'est mieux.

Enfin les pousser à la petite bande tout en gardant la dominante càd que
la 1 est plus près du milieu de la table que la 2 et la 3 : c'est le
pied !

* A partir de maintenant, ayons toujours en tête qu'il faut ramener les
billes à une petite bande et prendre la dominante. C'est le secret des
commencements de série. Nous ne pouvons pas déjà prétendre serrer le jeu
tout le temps ni que chaque position le permet mais le fait d'y penser,
de chercher, d'essayer, vos séries vont commencer à grandir... sans
qu'on ne s'en aperçoive. Et ne pensez pas à ce que vous laissez à
l'adversaire : si vous êtes plus fort, avec ou sans « jeu », vous
gagnerez plus souvent que vous ne perdrez... la réciproque étant vraie.

La figure : flèche un peu longue, canne légère mais ferme, main arrière
loin sur le talon et coup de rétro pas trop bas et de force soutenue
mais sans violence, effet à gauche. La 1 revient en direct sur la 3
lentement pendant que la 2 fait le tour de la table. A notre bonne
surprise, nous avons pris la dominante. Avec un peu de chance, le jeu se
poursuit par un coulé direct ou un 1B par l'arrière de la 3... et nous
voilà ramenés à la petite bande tout en gardant la dominante. C'est
merveilleux.

Remarques : - La prise grosse ramène la 2 haut.

- La prise fine la ramène basse.

- Suivant l'inclinaison 1-2, l'effet à droite redresse la 2 mais émousse
son énergie. → Nous avons la main mise sur le chemin que va emprunter la
2.

Il est préférable d'avoir la dominante même un peu difficile plutôt
qu'un serrage mal foutu !

\subsection{B22 - Rappel arrière par deux bandes -
}\label{b22---rappel-arriuxe8re-par-deux-bandes--}

Au début de cette année, nous avons déjà vu un deux bandes (B01).
Celui-ci fut expliqué à force de constructions géométriques et il
semblait bien compliqué. L'expérience nous a montré que la réalité du
terrain était moins difficile que la compréhension du dessin.

Nous voici avec le même deux bandes mais la 1 se trouve à l'intérieur du
jeu càd plus près du coin envisagé pour le calcul que les deux autres
billes. Nous n'avons donc pas les billes devant la 1 et nous n'avons pas
la dominante. Nous sommes apparemment dans une position défavorable, de
destruction...

Nous allons commencer par faire le même tracé

géométrique que pour la fiche B01. Apprécions le milieu du segment 2-3,
soit le point X. Traçons mentalement une droite passant par X et le
coin, soit la droite y. Traçons une parallèle à cette droite passant par
le centre de la 2 soit z. Cette parallèle coupe la bande au point R qui
devient notre point de repère rapproché. Il nous suffit maintenant de
jouer un coup direct en rétro, effet favorable normal...2-R-3 ... Pas si
vite ! Le calcul est exact mais l'effet induit en bas de bille est trop
puissant naturellement et à notre désappointement, la 1 va passer en bas
de la 3... Il sera donc indispensable d'appliquer une correction. Pour
cela, chacun a son truc... Je vous livre le mien quitte à choisir le
vôtre pourvu que ça marche. Cette parallèle qui détermine le point R,
avec la bille 2 pour centre, je la fais pivoter vers le haut de la
table, donc en s'éloignant du coin, d'un angle d'environ 15°...
déterminant ainsi un point de repère corrigé, soit Rc... Il nous reste à
appliquer le rétro normal, déjà cité plus haut, considérant maintenant
le point 2-Rc-3. Ne jouez pas trop fort : juste pour faire le chemin....
La 2 va faire le tour de la table et revenir gentiment dans le chapeau.

Remarque : Vous n'êtes pas tiré d'affaire car vous n'avez pas la
dominante mais le point est maintenant beaucoup plus facile et un jeu de
construction est envisageable. Celui qui maîtrise bien ce point a
l'avantage de réduire les risques d'un jeu éclaté.

·

Entraînement : Laissez la 3 comme sur la figure et faites varier la
hauteur de la 2 entre les mouches 1 et 3. Placez la 1 de manière à être
amené à devoir exécuter un angle supérieur, puis égal et enfin inférieur
à 90° en n'oubliant pas de prendre la 2 en rétro même en finesse s'il le
faut. Quand vous parviendrez à 100\% de réussite, votre moyenne montera
au moins de 1 La joie...

\subsection{B23 - Coulés en Cascade}\label{b23---couluxe9s-en-cascade}

Le coulé en cascade n'est pas un point en soi mais une tactique, une
méthode pour gagner le tiers du billard tout en gardant une position
dominante. Elle a pour but de ramener les billes près de la petite bande
en les laissant groupées.

Voyez la figure : au départ, nous sommes dans une position favorable de
construction mais au milieu du jeu de quilles. Examinez bien la
situation. Nous avons l'embarras du choix quant à la manière de jouer :

\begin{itemize}
\item
  \begin{quote}
  Tenter une finesse assure le point mais nous place non seulement au
  milieu des deux autres billes mais encore nous laisse au milieu de la
  table. Avec de la chance nous serons peut-être bien placés pour opérer
  un rétro réparateur et récupérer la dominante que nous aurons perdu :
  en général, ce sera une mauvaise solution.
  \end{quote}
\item
  \begin{quote}
  Si nous la « voyons », nous pouvons faire un rétro sur la 3. Nous
  prenons la dominante de l'autre côté de la table mais le serrage du
  jeu est loin d'être assuré. Cette solution est donc meilleure mais
  perfectible.
  \end{quote}
\end{itemize}

Un coulé est tellement plus prometteur... Position 1-2-3: Appliquez un
coulé simple. Evidemment, le jeu ne se serre pas d'un coup mais pour peu
qu'on maîtrise bien sa force et sa précision, l'ensemble se rapproche de
la petite bande et nous restons en dominante. Position 1'-2'-3',
résultat du coup précédent : re-coulé qui nous rapproche encore du fond
de la table en faisant tourner légèrement la figure. Position 1"-2"-3" :
c'est le coulé auquel on rêve, celui qui regroupe, qui serre. On profite
même du barrage près de la petite bande. C'est une promesse de série qui
s'annonce.

Remarque : Expliqué de la sorte, nous sommes dans le rêve, dans l'idéal.
La réalité sera souvent plus prosaïque et la dentelle figurée moins
régulière. On devra parfois jouer plus dur, ou plus amorti ou même
transformer un coulé en rétro. Il conviendra de bien choisir à chaque
étape. L'important est de garder le principe en tête.

Entraînement : Placer les billes en position de construction comme le
montre la figure et essayer d'exécuter la cascade jusqu'à ce que le
raisonnement sur papier devienne une mécanique dans les mains. Ce sera
gagné lorsque vous parviendrez à amener les 3 billes à la petite bande
dans un chapeau tout en gardant la dominante. Bon courage...

\subsection{B24 - Rétro dans le
tiers}\label{b24---ruxe9tro-dans-le-tiers}

Ayez toujours dans la tête qu'il faut rabattre les billes à la bande si
on veut progresser dans le jeu de série. Il est préférable de jouer
rabattu à la grande bande plutôt que dans « tout » le billard et il est
préférable de rabattre à la petite bande plutôt qu'à la grande. Amener
les billes dans le « bas » du billard est ce qu'on appelle « jouer dans
le tiers ». Plus tard, nous verrons qu'il est encore plus « fin »
d'amener le jeu sur sa « bonne main » càd celle qui permet un
positionnement du corps sans problème. Par exemple, voyez la figure 1.
Nous sommes amené à bricoler d'un côté ou l'autre selon qu'on est
gaucher ou droitier. Vous constaterez par vous-même que si vous avez mal
choisi, votre corps se heurtera à la table au coup suivant. Par contre
si votre choix est judicieux, au coup suivant, votre corps sera
parfaitement dégagé.

Figure 2: D'apparence facile, ce point demande une bonne précision.
Réalisé parfaitement, il assure une possibilité de serrage à la petite
bande en quelques coups et constitue même un point de base au cadre car
il permet aussi un serrage à la ligne. En libre, l'important sera de
veiller à garder la dominante puis d'écraser le jeu sur la petite bande.
Appliquons donc un simple rétro en veillant à garder la dominante sur la
3. La force du coup doit ramener la 2 dans la vision du jeu mais plus
proche de la petite bande. Si le rétro est court, il conviendra de le
durcir, sinon, s'il est long, de l'assouplir. L'expérience nous le fera
appliquer systématiquement. Comme le montre l'exemple de la figure, la 2
tourne dans le fond de la table et revient dans le champ inférieur du
jeu. Si 1-2 est peu incliné, favorisez le « tour
\textgreater{}\textgreater{} en ajoutant du bon effet, trop penché, du
contraire. Pour un bon retour de la 2, il est préférable qu'elle touche
la petite bande en deuxième bande mais si cela n'était pas possible, il
est préférable de

dominante plutôt que forcer la petite bande à la 2, pourvu qu'elle reste
dans le bas du jeu.

coulé ou un rétro sur la 3 (devenue la 2) dans le même état d'esprit.
Que le premier coup. Si par malheur, le jeu se retourne et que l'on
perde la dominante, n'ayez pas peur de repousser le jeu dans le haut de
la table et de recommencer la même tactique. Rien que chercher la
dominante et de ramener dans le tiers va vous amener des séries...

Remarque de fin de deuxième année : beaucoup de joueurs se plaignent
d'un billard « qui ne roule pas ».... Il est vrai que la texture même du
revêtement, la poussière accumulée dans le tapis, les billes sales,
froides ou humides sont autant de facteurs qui durcissent le jeu.
Cependant, les joueurs plaignants adoptent la résolution de serrer la
canne avec la main arrièr de la table... C'est le contraire qu'il faut
faire : soyez souple et pénétrant, amenez la légèreté pour combattre la
lourdeur...
